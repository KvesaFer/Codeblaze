\chapter{Opis projektnog zadatka}
		
		Cilj ovog projekta je razviti programsku podršku za stvaranje web aplikacije „Iznajmi romobil“ koja će korisnicima omogućiti da iznajme svoj električni romobil u periodima dana kada ga ne koriste. Aplikacija će korisnicima omogućavati brz i jednostavan pristup električnim romobilima dostupnima za najam kao i postavljanje ponude za iznajmljivanje svog romobila. Električni romobili su danas odlična alternativa za automobile te zbog svoje ekološke prihvatljivosti i praktičnosti postaju sve popularniji u većim urbanim sredinama. Osoba koja želi iznajmiti svoj romobil, odnosno iznajmljivač, prilikom registracije romobila, unosi osnovne informacije o romobili. Kada je romobil registriran, iznajmljivač može objaviti oglas za isti pri čemu upisuju gdje se romobil trenutno nalazi te gdje i kada mora biti vraćen. Osoba koja želi unajmiti romobil, odnosno klijent na temelju ponuđenih romobila i informacija o njima odabire onaj koji je u tom trenutku dostupan i najviše odgovara njegovim potrebama. 
		\newline 
		\newline
		Prilikom pokretanja aplikacije, korisnicima se neovisno o tome jesu li prijavljeni ili ne prikazuje popis svih aktivnih oglasa romobila. Uz svaki su romobil navedene njegove karakteristike, cijena, lokacija na kojoj se romobil nalazi te vrijeme i lokacija gdje romobil mora biti vraćen. \textbf{Neregistrirani korisnici} mogu pregledavati trenutno dostupne romobile i njihova cijene, ali ih ne mogu iznajmiti. Nakon što kreiraju novi korisnički račun, ponuđeni romobili im postaju dostupni za najam. Prilikom kreiranja novog računa korisnici moraju unijeti sljedeće podatke:
		
		\begin{packed_item}
			\item \textit{ime i prezime}
			\item \textit{email adresa}
			\item \textit{nadimak}
			\item \textit{broj kartice}
		\end{packed_item}
		
		Osim navedenog, korisnici prilikom registracije moraju dostaviti kopiju osobne iskaznice i potvrdu o nekažnjavanju. Nakon što su uneseni svi podaci i dostavljeni potrebni dokumenti, korisnik čeka pregled dokumenata odnosno odobravanje ili odbijanje registracije. Dok registracija nije odobrena, korisnik se ne može prijaviti u sustav. Ako je zahtjev za registraciju odbijen zbog neispravnosti dostavljenih dokumenata, korisnik može ponovno predati dokumente. Korisnici koji su unaprijed registrirani, mogu se prijaviti u sustav sa svojim postojećim korisničkim računom tako da unesu email adresu i lozinku. Ako korisnik prilikom unosa podataka za prijavu unese podatke koji ne odgovaraju nijednom registriranom korisniku u bazi, šalje mu se obavijest o neispravnosti podataka. \textbf{Klijent} je korisnik aplikacije koji može pregledavati i unajmljivati romobile, ali nema registriranih vlastitih romobila. Kada odabere romobil koji želi iznajmiti, klijent se javlja na oglas s automatski generiranom porukom koja sadrži zahtjev za iznajmljivanje. Nakon što je ponuda prihvaćena, klijentu se šalje obavijest da je iznajmljivanje odobreno i oglas se briše. Klijent prije pokretanja romobila provjerava odgovara li fotografija romobila njegovom stvarnom stanju. Ako ne odgovara, on ima mogućnost odabirom opcije "Zamijeni sliku" zamijeniti sliku romobila novom slikom i kratkim opisom o razlikama između nove i stare slike. U slučaju da se iznajmljivač žali na zamjenu slika, klijent biva obavješten o odluci. Na kraju iznajmljivanja, klijent vraća romobil i u aplikaciji potvrđuje da ga je vratio. Nakon toga slijedi provjera je li romobil vraćen u pravo vrijeme te se izračunava cijena koju klijent mora platiti. Svaki klijent u aplikaciji može pregledati svoj profil na kojem se nalaze njegovi osobni podaci. Za sve podatke on sam određuje hoće li oni biti javno dostupni ili ne, a iste može urediti odabirom opcije "Uredi profil".  Pri uređivanju profila provjerava se jesu li novi podatci uneseni u ispravnom formatu, ako nisu korisnik dobiva obavijest o neispravnosti. Nakon unosa promjena, klijent mora odabrati opciju "Spremi promjene" kako bi potvrdio pohranjivanje promjena u bazu podataka. Na profilu svakog klijenta moraju biti vidljive ocjene i komentari iznajmljivača. Klijent, u trenutku kada registrira svoj romobil, postaje \textbf{iznajmljivač}. Prilikom registracije romobila potrebno je unijeti sljedeće podatke o romobilu: 
		
			\begin{packed_item}
			\item \textit{naziv proizvođača}
			\item \textit{naziv modela}
			\item \textit{kapacitet baterije}
			\item \textit{maksimalna brzina}
			\item \textit{URL fotografije romobila}
			\item \textit{maksimalni domet}
			\item \textit{godinu proizvodnje}
			\item \textit{dostupnost}
			\item \textit{dodatne informacije (po želji)}
			\end{packed_item}
			
		Prilikom postavljanja ponude za iznajmljivanje, iznajmljivač unosi trenutnu lokaciju romobila, lokaciju na koju želi da se romobil vrati, vrijeme do kada romobil mora biti vraćen, cijenu iznajmljivanja po prijeđenom kilometru te iznos novčane kazne u slučaju da romobil ne bude vraćen na vrijeme. Ako je neki romobil dostupan za iznajmljivanje, iznajmljivač oglas može podijeliti i na nekoj društvenoj mreži odabirom opcije "Objavi na društvenu mrežu". Svaki iznajmljivač može pregledavati svoje registrirane romobile, brisati postojeće i dodavati nove. Ako iznajmljivač izbriše sve svoje registrirane romobile, ono ponovno postaje klijent. Unutar aplikacije dostupna je mogućnost izmjenjivanja poruka kako bi se klijent i iznajmljivač mogli dogovoriti oko najma. Iznajmljivač pregledava zahtjeve za iznajmljivanje i profile klijenata te tada može prihvatiti ili odbiti ponudu. Ako klijent zamijeni sliku romobila jer smatra da ne prikazuje stvarno stanje romobila, iznajmljivaču se šalje obavijest o zamjeni. On tada, ako smatra da klijentova fotografija nije ispravna, može poslati žalbu na zamjenu slika. Nakon što je donesena odluka o zamjeni slika, klijentu i iznajmljivaču se šalje obavijest o donesenoj odluci. Na kraju iznajmljivanja, iznajmljivač kao i klijent dobiva obavijest da je iznajmljivanje završeno i cijenu iznajmljivanja koju mu klijent treba platiti. Iznajmljivač nakon završetka iznajmljivanja može ocijeniti klijenta i napisati komentar. \textbf{Administrator} pregledava dokumente dostavljene prilikom registracije te odbija ili prihvaća zahtjeve za registraciju. Osim toga, on zaprima žalbe na zamjenu slika romobila. Nakon zaprimanja žalbe administrator pregledava slike i odabire onu koja će se pohraniti u bazu. Administrator  ima pravo blokirati korisnika odnosno zabraniti mu pristup aplikaciji odabirom opcije "Blokiraj korisnika". Za takvog se korisnika u bazu podataka upisuje da je blokiran te će mu pri sljedećoj prijavi biti onemogućen pristup sustavu. 
		\newline
		\newline
		
		
		
		
		
		
		
	